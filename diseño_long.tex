% Options for packages loaded elsewhere
% Options for packages loaded elsewhere
\PassOptionsToPackage{unicode}{hyperref}
\PassOptionsToPackage{hyphens}{url}
\PassOptionsToPackage{dvipsnames,svgnames,x11names}{xcolor}
%
\documentclass[
  spanish,
  13pt,
]{article}
\usepackage{xcolor}
\usepackage[margin=2cm]{geometry}
\usepackage{amsmath,amssymb}
\setcounter{secnumdepth}{-\maxdimen} % remove section numbering
\usepackage{iftex}
\ifPDFTeX
  \usepackage[T1]{fontenc}
  \usepackage[utf8]{inputenc}
  \usepackage{textcomp} % provide euro and other symbols
\else % if luatex or xetex
  \usepackage{unicode-math} % this also loads fontspec
  \defaultfontfeatures{Scale=MatchLowercase}
  \defaultfontfeatures[\rmfamily]{Ligatures=TeX,Scale=1}
\fi
\usepackage{lmodern}
\ifPDFTeX\else
  % xetex/luatex font selection
  \setmainfont[]{Times New Roman}
\fi
% Use upquote if available, for straight quotes in verbatim environments
\IfFileExists{upquote.sty}{\usepackage{upquote}}{}
\IfFileExists{microtype.sty}{% use microtype if available
  \usepackage[]{microtype}
  \UseMicrotypeSet[protrusion]{basicmath} % disable protrusion for tt fonts
}{}
\usepackage{setspace}
\makeatletter
\@ifundefined{KOMAClassName}{% if non-KOMA class
  \IfFileExists{parskip.sty}{%
    \usepackage{parskip}
  }{% else
    \setlength{\parindent}{0pt}
    \setlength{\parskip}{6pt plus 2pt minus 1pt}}
}{% if KOMA class
  \KOMAoptions{parskip=half}}
\makeatother
% Make \paragraph and \subparagraph free-standing
\makeatletter
\ifx\paragraph\undefined\else
  \let\oldparagraph\paragraph
  \renewcommand{\paragraph}{
    \@ifstar
      \xxxParagraphStar
      \xxxParagraphNoStar
  }
  \newcommand{\xxxParagraphStar}[1]{\oldparagraph*{#1}\mbox{}}
  \newcommand{\xxxParagraphNoStar}[1]{\oldparagraph{#1}\mbox{}}
\fi
\ifx\subparagraph\undefined\else
  \let\oldsubparagraph\subparagraph
  \renewcommand{\subparagraph}{
    \@ifstar
      \xxxSubParagraphStar
      \xxxSubParagraphNoStar
  }
  \newcommand{\xxxSubParagraphStar}[1]{\oldsubparagraph*{#1}\mbox{}}
  \newcommand{\xxxSubParagraphNoStar}[1]{\oldsubparagraph{#1}\mbox{}}
\fi
\makeatother


\usepackage{longtable,booktabs,array}
\usepackage{calc} % for calculating minipage widths
% Correct order of tables after \paragraph or \subparagraph
\usepackage{etoolbox}
\makeatletter
\patchcmd\longtable{\par}{\if@noskipsec\mbox{}\fi\par}{}{}
\makeatother
% Allow footnotes in longtable head/foot
\IfFileExists{footnotehyper.sty}{\usepackage{footnotehyper}}{\usepackage{footnote}}
\makesavenoteenv{longtable}
\usepackage{graphicx}
\makeatletter
\newsavebox\pandoc@box
\newcommand*\pandocbounded[1]{% scales image to fit in text height/width
  \sbox\pandoc@box{#1}%
  \Gscale@div\@tempa{\textheight}{\dimexpr\ht\pandoc@box+\dp\pandoc@box\relax}%
  \Gscale@div\@tempb{\linewidth}{\wd\pandoc@box}%
  \ifdim\@tempb\p@<\@tempa\p@\let\@tempa\@tempb\fi% select the smaller of both
  \ifdim\@tempa\p@<\p@\scalebox{\@tempa}{\usebox\pandoc@box}%
  \else\usebox{\pandoc@box}%
  \fi%
}
% Set default figure placement to htbp
\def\fps@figure{htbp}
\makeatother


% definitions for citeproc citations
\NewDocumentCommand\citeproctext{}{}
\NewDocumentCommand\citeproc{mm}{%
  \begingroup\def\citeproctext{#2}\cite{#1}\endgroup}
\makeatletter
 % allow citations to break across lines
 \let\@cite@ofmt\@firstofone
 % avoid brackets around text for \cite:
 \def\@biblabel#1{}
 \def\@cite#1#2{{#1\if@tempswa , #2\fi}}
\makeatother
\newlength{\cslhangindent}
\setlength{\cslhangindent}{1.5em}
\newlength{\csllabelwidth}
\setlength{\csllabelwidth}{3em}
\newenvironment{CSLReferences}[2] % #1 hanging-indent, #2 entry-spacing
 {\begin{list}{}{%
  \setlength{\itemindent}{0pt}
  \setlength{\leftmargin}{0pt}
  \setlength{\parsep}{0pt}
  % turn on hanging indent if param 1 is 1
  \ifodd #1
   \setlength{\leftmargin}{\cslhangindent}
   \setlength{\itemindent}{-1\cslhangindent}
  \fi
  % set entry spacing
  \setlength{\itemsep}{#2\baselineskip}}}
 {\end{list}}
\usepackage{calc}
\newcommand{\CSLBlock}[1]{\hfill\break\parbox[t]{\linewidth}{\strut\ignorespaces#1\strut}}
\newcommand{\CSLLeftMargin}[1]{\parbox[t]{\csllabelwidth}{\strut#1\strut}}
\newcommand{\CSLRightInline}[1]{\parbox[t]{\linewidth - \csllabelwidth}{\strut#1\strut}}
\newcommand{\CSLIndent}[1]{\hspace{\cslhangindent}#1}

\ifLuaTeX
\usepackage[bidi=basic]{babel}
\else
\usepackage[bidi=default]{babel}
\fi
\ifPDFTeX
\else
\babelfont{rm}[]{Times New Roman}
\fi
% get rid of language-specific shorthands (see #6817):
\let\LanguageShortHands\languageshorthands
\def\languageshorthands#1{}


\setlength{\emergencystretch}{3em} % prevent overfull lines

\providecommand{\tightlist}{%
  \setlength{\itemsep}{0pt}\setlength{\parskip}{0pt}}



 


\usepackage[noblocks]{authblk}
\renewcommand*{\Authsep}{, }
\renewcommand*{\Authand}{, }
\renewcommand*{\Authands}{, }
\renewcommand\Affilfont{\small}
\makeatletter
\@ifpackageloaded{caption}{}{\usepackage{caption}}
\AtBeginDocument{%
\ifdefined\contentsname
  \renewcommand*\contentsname{Tabla de contenidos}
\else
  \newcommand\contentsname{Tabla de contenidos}
\fi
\ifdefined\listfigurename
  \renewcommand*\listfigurename{Listado de Figuras}
\else
  \newcommand\listfigurename{Listado de Figuras}
\fi
\ifdefined\listtablename
  \renewcommand*\listtablename{Listado de Tablas}
\else
  \newcommand\listtablename{Listado de Tablas}
\fi
\ifdefined\figurename
  \renewcommand*\figurename{Figura}
\else
  \newcommand\figurename{Figura}
\fi
\ifdefined\tablename
  \renewcommand*\tablename{Tabla}
\else
  \newcommand\tablename{Tabla}
\fi
}
\@ifpackageloaded{float}{}{\usepackage{float}}
\floatstyle{ruled}
\@ifundefined{c@chapter}{\newfloat{codelisting}{h}{lop}}{\newfloat{codelisting}{h}{lop}[chapter]}
\floatname{codelisting}{Listado}
\newcommand*\listoflistings{\listof{codelisting}{Listado de Listados}}
\makeatother
\makeatletter
\makeatother
\makeatletter
\@ifpackageloaded{caption}{}{\usepackage{caption}}
\@ifpackageloaded{subcaption}{}{\usepackage{subcaption}}
\makeatother
\usepackage{bookmark}
\IfFileExists{xurl.sty}{\usepackage{xurl}}{} % add URL line breaks if available
\urlstyle{same}
\hypersetup{
  pdftitle={Diseño estadístico: Encuesta sobre las Implicaciones Psicosociales de la Pobreza, Riqueza y Desigualdad Socioeconómica (PORIDE)},
  pdfauthor={Mario Sainz Martínez},
  pdflang={es},
  colorlinks=true,
  linkcolor={blue},
  filecolor={Maroon},
  citecolor={Blue},
  urlcolor={Blue},
  pdfcreator={LaTeX via pandoc}}


\title{Diseño estadístico: Encuesta sobre las Implicaciones
Psicosociales de la Pobreza, Riqueza y Desigualdad Socioeconómica
(PORIDE)}


  \author{Mario Sainz Martínez}
            \affil{%
                  Departamento de Psicología Social, Facultad de
                  Psicología, Universidad de Granada, Spain
              }
      
\date{}
\begin{document}
\maketitle


\setstretch{1.5}
\section{Presentación}\label{presentaciuxf3n}

Esta propuesta expone el diseño estadístico para la implementación de
una encuesta longitudinal dirigida a personas en situación de pobreza en
España. Para su elaboración se recurrió a tres fuentes principales: i)
los textos de referencia de Groves et~al.
(\citeproc{ref-groves_survey_2011}{2011}), Survey Methodology, y de Lynn
(\citeproc{ref-lynn_methodology_2009}{2009}), Methodology of
Longitudinal Surveys, que ofrecen los fundamentos teóricos y técnicos en
la construcción de encuestas y estudios longitudinales; ii) el diseño de
la \href{https://manual-metodologico-elsoc.netlify.app/}{Encuesta
Longitudinal Social de Chile (ELSOC)} desarrollada por el COES,
utilizada como modelo comparativo y de orientación metodológica; y iii)
los datos de la
\href{https://www.ine.es/dyngs/INEbase/es/operacion.htm?c=Estadistica_C&cid=1254736176807&menu=resultados&secc=1254736195153&idp=1254735976608\#_tabs-1254736195153}{Encuesta
de Condiciones de Vida del INE} en España, que proporcionan información
clave sobre la magnitud y características de la población en situación
de pobreza según ingresos. A partir de estas referencias, el documento
presenta los elementos esenciales para estructurar un instrumento
longitudinal enfocado en el estudio de esta población específica.

\section{Supuestos}\label{supuestos}

Para alinear la propuesta con criterios operativos realistas y control
explícito del Error Total de Encuesta (TSE), adoptamos los siguientes
supuestos y metas de calidad (\citeproc{ref-groves_survey_2011}{Groves
et~al., 2011}; \citeproc{ref-lynn_methodology_2009}{Lynn, 2009}):

\begin{itemize}
\tightlist
\item
  \textbf{Población objetivo:} Personas de 16+ años en \textbf{pobreza
  por ingresos} en España (ámbito nacional), definidas al momento del
  muestreo. La población de referencia se fija en Ola 1; las mismas
  personas se siguen en Olas 2 y 3 (panel fijo.
\item
  \textbf{Estrategia muestral:} \textbf{Muestreo aleatorio simple (SRS)}
  de individuos, \textbf{sin estratificación} ni \textbf{conglomeración}

  \begin{itemize}
  \tightlist
  \item
    \textbf{Implicación:} \(Deff = 1\) como supuesto operativo (no hay
    inflación de varianza por diseño).
  \item
    \textbf{Nota:} si los ajustes por no respuesta/calibración generan
    variabilidad de pesos, se monitorizará el \textbf{deff por pesos}
    para contener pérdidas de eficiencia
    (\citeproc{ref-groves_survey_2011}{Groves et~al., 2011}).
  \end{itemize}
\item
  \textbf{Dominios de estimación:} Publicación en \textbf{Total
  nacional} y \textbf{Sexo} (Mujeres/Hombres). No se exige precisión por
  CCAA en este diseño.
\item
  \textbf{Precisión objetivo:} Margen de error \(\le \pm 5\) p.p., 95\%
  CI, peor caso \(p=0.5\):\\
  \[
  n_{\text{ME}} \;=\; \frac{z_{0.975}^2\,p(1-p)}{e^2}
  \;=\; \frac{1.96^2\cdot 0.25}{0.05^2}
  \;\approx\; 385
  \] Para garantizar \(\pm 5\) p.p. \textbf{por sexo} en \textbf{ola 3},
  el tamaño total \(n_3\) debe satisfacer\\
  \(n_3\cdot \pi_f \ge 385 \;\wedge\; n_3\cdot \pi_m \ge 385\),\\
  con \(\pi_f \approx 0.528\) y \(\pi_m \approx 0.470\) (proporciones
  observadas en la población pobre).
\item
  \textbf{Poder estadístico (ola 3):} \textbf{\(\ge 0.80\)} para
  detectar \textbf{\(\delta \approx 8\) p.p.} de diferencia por sexo,
  con proporciones cercanas a \(0.5\) y prueba bilateral
  \(\alpha=0.05\):\\
  \[
  n_{\text{por\,sexo}} \;\approx\;
  \frac{2\,\bar p(1-\bar p)\,\left(z_{1-\alpha/2}+z_{1-\beta}\right)^2}{\delta^2}
  \;=\; \frac{2\cdot 0.25\cdot (1.96+0.84)^2}{0.08^2}
  \;\approx\; 613
  \] \textbf{Meta conservadora:} fijamos ola 3 total en
  \(n_3 \approx 1{,}226\) (≈ \(613\) por sexo), lo que cumple poder
  \(\ge 0.80\) y supera el mínimo de precisión (\(385\)/sexo).
\item
  \textbf{Atrición y horizonte longitudinal:} Tres olas, sin refresco,
  con atrición total del 30\% entre Ola 1 y Ola 3.

  \begin{itemize}
  \tightlist
  \item
    \textbf{Retención acumulada (O1→O3):} \(0.70\).
  \item
    \textbf{Retención por intervalo (O1→O2 y O2→O3):}
    \(r=\sqrt{0.70}\approx 0.8367\).
  \end{itemize}
\item
  \textbf{Dimensionamiento por ola (consistente con poder y
  precisión):}\\
  \[
  n_{1}=\left\lceil \frac{n_{3}}{0.70}\right\rceil,\quad
  n_{2}\approx \left\lceil n_{1}\cdot r\right\rceil,\quad
  n_{3}\approx \left\lceil n_{2}\cdot r\right\rceil
  \] Con \(n_3 \approx 1{,}226 \Rightarrow\) \(n_1 \approx 1{,}752\),
  \(n_2 \approx 1{,}466\), \(n_3 \approx 1{,}226\).
\item
  \textbf{Ponderación y control de sesgo:} Peso base SRS; ajuste por no
  respuesta (clases o propensión) y calibración a \textbf{Total} y
  \textbf{Sexo} por ola. El sesgo de no respuesta depende de
  \(\mathrm{cov}(y,p)\) (no de la tasa bruta); se monitorizará junto con
  el deff por pesos.
\end{itemize}

\section{Resumen numérico derivado:}\label{resumen-numuxe9rico-derivado}

\begin{itemize}
\tightlist
\item
  \textbf{Ola 1 = 1,752};
\item
  \textbf{Ola 2 = 1,466};
\item
  \textbf{Ola 3 = 1,226} (≈ \(613\) por sexo).
\item
  \textbf{Total ≈ 4,444 entrevistas}
\end{itemize}

En ola 3 se garantiza \textbf{ME \(\le \pm 5\) p.p.} por sexo y
\textbf{poder \(\ge 0.80\)} para \(\delta \approx 8\) p.p.

\section{Universo}\label{universo}

El universo del estudio serán los hombres y mujeres que ese encuentren
en situación de pobreza en España en 2026.

\section{Población objetivo}\label{poblaciuxf3n-objetivo}

La población objetivo serán los hombres y mujeres mayores de 18 años
residentes en hogares particulares en España que se encuentran en
situación de pobreza de acuerdo con el indicador acordado (p.~ej., AROPE
estimado desde la ECV/INE) en 2026. La población de referencia se define
en la Ola 1 y las mismas personas son seguidas en dos olas adicionales
(panel de tres olas anuales).

\section{Unidad de muestreo}\label{unidad-de-muestreo}

La unidad de muestreo son los hombres y mujeres (individuos) mayores de
18 años que se encuentran en situación de pobreza por ingresos en
seleccionados directamente en la población.

\section{Unidad de análisis y de
información}\label{unidad-de-anuxe1lisis-y-de-informaciuxf3n}

La unidad de análisis serán los individuos en situación de pobreza
cumplen los requisitos para formar parte de la población objetivo. En
cada ola se levanta un módulo de hogar (composición e ingresos) para
reconstituir la situación de pobreza del entrevistado.

La unidad de información serán los hombres y mujeres en situación de
pobreza mayores de 18 años que cumplen los requisitos para formar parte
de la población objetivo y que resulten seleccionadas en el diseño
muestral.

Movilidad (movers): se siguen las personas muestrales allí donde residan
(definir reglas operativas). Lynn
(\citeproc{ref-lynn_methodology_2009}{2009}) destaca la necesidad de
reglas frente a cambios de unidad (p.~ej., mudanzas, composición del
hogar).

\section{Marco muestral}\label{marco-muestral}

El marco muestral corresponde a un listado o procedimientos que permiten
identificar todos los elementos con probabilidad de ser seleccionados de
la población objetivo (\citeproc{ref-groves_survey_2011}{Groves et~al.,
2011}). Se asume muestreo probabílistico aleatorio simple (SRS)
individual y cobertura nacional (sin dominios territoriales).

\subsection{Conformación marco
muestral}\label{conformaciuxf3n-marco-muestral}

Completar. Para esto se usa info de estadisticas oficiales de cantidad y
proporcion de hombres y mujeres en pobreza a nivel nacional. Requisito:
un marco nominal de personas en pobreza (p.~ej., registros
administrativos integrados y depurados o listados derivados de un
operativo de cribado).

\subsection{Selección de unidades}\label{selecciuxf3n-de-unidades}

\begin{enumerate}
\def\labelenumi{\arabic{enumi}.}
\tightlist
\item
  Construcción/obtención de listado de elegibles (personas en pobreza
  por ingresos).
\item
  Persona: selección aleatoria simple de \(n\) individuos a nivel
  nacional para Ola 1.
\item
  Seguimiento longitudinal: reentrevistar a los mismos individuos en Ola
  2 y Ola 3. Corresponde a un panel fijo de tres olas (\(t_0\), \(t_1\),
  \(t_2\)) sin refresco, atrición esperada 30\% por ola (r≈0.70 por
  transición).
\end{enumerate}

\section{Diseño muestral
longitudinal}\label{diseuxf1o-muestral-longitudinal}

\begin{itemize}
\tightlist
\item
  Tipo de muestreo: Probabilístico aleatorio simple (SRS) de personas en
  el marco muestral. No hay estratos ni conglomerados: esto simplifica
  el peso base y elimina deff por conglomeración; a costa de no
  garantizar auto-ponderación por subgrupos geográficos\footnote{Cuando
    se usan estratos o asignaciones desproporcionadas, pueden lograrse
    ganancias de precisión si la estratificación reduce la varianza
    interna; aquí se renuncia a esa ganancia para simplificar y
    priorizar poder por sexo, manteniendo SRS (deff≈1)
    (\citeproc{ref-groves_survey_2011}{Groves et~al., 2011}).}.
\item
  Panel de tres olas (\(t_0\), \(t_1\), \(t_2\)) reentrevistando a los
  mismos individuos, sin refresco, atrición esperada 30\% por ola
  (r≈0.70 por transición).
\item
  Sin refresco: cualquier pérdida entre olas se refleja en menor tamaño
  en la ola siguiente.
\item
  Plan de retención: protocolos de contacto, tracking de ``movers'',
  modos de seguimiento flexibles y recordatorios para minimizar
  atrición.
\end{itemize}

\subsection{Dominio de estudio}\label{dominio-de-estudio}

\begin{itemize}
\tightlist
\item
  Total nacional.
\item
  Sexo: Mujeres y Hombres (dominios principales de publicación y
  precisión).
\end{itemize}

\section{Cálculo y distribución del tamaño
muestral}\label{cuxe1lculo-y-distribuciuxf3n-del-tamauxf1o-muestral}

\subsection{Parámetros poblacionales
conocidos}\label{paruxe1metros-poblacionales-conocidos}

\begin{itemize}
\tightlist
\item
  Total personas en pobreza: \(N = 9{,}578{,}080\).
\item
  Mujeres: \(5{,}057{,}736\Rightarrow \pi_f\approx 0.528\).
\item
  Hombres: \(4{,}503{,}279\Rightarrow \pi_m\approx 0.470\).
\end{itemize}

\subsection{Precisión objetivo en ola
3}\label{precisiuxf3n-objetivo-en-ola-3}

Bajo SRS, \(p=0.5\) y 95\% CI:

\[
n_{\text{ME}}=\frac{z_{0.975}^2\,p(1-p)}{e^2}
=\frac{1.96^2\cdot 0.25}{0.05^2}\approx 384.16\;\Rightarrow\;385.
\]

Para garantizar \textbf{por sexo} en \textbf{ola 3}:

\[
n_{3}\times \pi_f \ge 385,\qquad n_{3}\times \pi_m \ge 385
\Rightarrow n_{3}\ge \max\!\Big(\frac{385}{\pi_f},\,\frac{385}{\pi_m}\Big)
= \frac{385}{0.470}\approx \mathbf{822}.
\]

\subsection{Poder estadístico}\label{poder-estaduxedstico}

Poder ≥ 0.80 para \(δ≈8\) p.p. (sexo, ola 3). Escenario conservador
(proporciones cercanas a 0.5, tamaños similares por sexo), prueba
bilateral \((alpha=0.05)\):

\[
n_{\text{por sexo}}\approx 
\frac{2\,\bar{p}(1-\bar{p})\,(z_{1-\alpha/2}+z_{1-\beta})^2}{\delta^2}
=\frac{2\cdot 0.25\cdot (1.96+0.84)^2}{0.08^2}
\approx \mathbf{613}.
\]

Por lo tanto, fijamos \textbf{ola 3 total} en:

\[
n_{3} \approx \mathbf{1{,}226}\quad (\approx 613 \text{ por sexo}),
\]

que \textbf{cumple poder ≥ 0.80} para δ≈8 p.p. y \textbf{supera} el
mínimo de precisión (385/sexo).

\subsection{Dimensionamiento por ola (sin refresco; atrición total
30\%)}\label{dimensionamiento-por-ola-sin-refresco-atriciuxf3n-total-30}

Con \textbf{retención acumulada} \((=0.70)\) y \textbf{retención por
intervalo} \((r=\sqrt{0.70}\approx 0.8367)\):

\[
n_{1}=\left\lceil \frac{n_{3}}{0.70} \right\rceil
=\left\lceil \frac{1{,}226}{0.70}\right\rceil
=\mathbf{1{,}752},
\]

\[
n_{2}\approx \left\lceil n_{1}\cdot r \right\rceil
=\left\lceil 1{,}752\cdot 0.8367\right\rceil
=\mathbf{1{,}466},
\]

\[
n_{3}\approx \left\lceil n_{2}\cdot r \right\rceil
=\left\lceil 1{,}466\cdot 0.8367\right\rceil
=\mathbf{1{,}226}.
\]

\subsubsection{\texorpdfstring{Tabla de tamaños (SRS, sin refresco,
atrición total
30\%){[}\textsuperscript{2,}3{]}}{Tabla de tamaños (SRS, sin refresco, atrición total 30\%){[}2,3{]}}}\label{tabla-de-tamauxf1os-srs-sin-refresco-atriciuxf3n-total-3023}

\begin{longtable}[]{@{}rrrr@{}}
\toprule\noalign{}
Ola & Total esperado & Mujeres (≈52.8\%) & Hombres (≈47.0\%) \\
\midrule\noalign{}
\endhead
\bottomrule\noalign{}
\endlastfoot
\textbf{Ola 1} & \textbf{1,752} & ≈ 925 & ≈ 823 \\
\textbf{Ola 2} & \textbf{1,466} & ≈ 774 & ≈ 689 \\
\textbf{Ola 3} & \textbf{1,226} & \textbf{≈ 647} & \textbf{≈ 576} \\
\end{longtable}

\begin{itemize}
\tightlist
\item
  En ola 3, ambas celdas superan 613/sexo si la composición observada se
  aproxima a \(\pi_f,\pi_m\).
\item
  Si se desea exactamente 613 por sexo (sin depender de proporciones),
  puede estratificarse por sexo con asignación igual, manteniendo SRS
  dentro de cada estrato.
\end{itemize}

\subsubsection{Verificación rápida}\label{verificaciuxf3n-ruxe1pida}

Precisión en ola 3 (SRS):\\
- Mujeres \((n\approx 647)\) \(\Rightarrow\)
\((e\approx 1.96\sqrt{0.25/647}\approx 3.85)\) p.p.\\
- Hombres \((n\approx 576)\) \(\Rightarrow\)
\((e\approx 1.96\sqrt{0.25/576}\approx 4.09)\) p.p.\\
- (Ambos \textbf{mejores} que ±5 p.p.)

\textbf{Poder (δ≈8 p.p.):} \((n_{\text{por sexo}}\ge 613)\) garantiza
\((1-\beta\ge 0.80)\) en el peor caso; con
\((n_f\approx647, n_m\approx576)\), el poder es \textbf{≥ 0.80}.

\section{Atrición (prevención y
medición)}\label{atriciuxf3n-prevenciuxf3n-y-mediciuxf3n}

\begin{itemize}
\tightlist
\item
  \textbf{Prevención:} protocolos de recontacto, \emph{tracking} de
  movers, incentivos, modos mixtos en seguimiento, calendario flexible.
\item
  \textbf{Medición de cambio:} reducir \emph{seam bias} y cambios de
  medición con \textbf{instrumentos consistentes} y \textbf{dependent
  interviewing} cuando proceda
  (\citeproc{ref-lynn_methodology_2009}{Lynn, 2009}).
\item
  \textbf{Sesgo de no respuesta:} depende de
  \(\operatorname{cov}(y, p)\), no de la tasa bruta; monitorizar perfil
  de respondedores vs.~no respondedores
  (\citeproc{ref-groves_survey_2011}{Groves et~al., 2011}).
\end{itemize}

\section{Factores de expansión
(ponderadores)}\label{factores-de-expansiuxf3n-ponderadores}

\subsection{Peso base}\label{peso-base}

Con \textbf{SRS}:

\[
w_i^{(0)}=\frac{N_M}{n_{1}},
\]

donde \((N_M)\) es el tamaño del \textbf{marco} de personas en pobreza.

\subsection{Ajuste por no respuesta (por
ola)}\label{ajuste-por-no-respuesta-por-ola}

Clases de ajuste o \textbf{modelo de propensión} con variables
observables del marco (sexo, edad, historial de contacto, etc.):

\[
w_i^{(\text{NR},t)}=\frac{w_i^{(t-1)}}{\hat{p}_{g(i),t}},
\]

con \(g(i)\) clases homogéneas o \(\hat{p}\) de un modelo (logit).
Estándar recomendado en TSE (\citeproc{ref-groves_survey_2011}{Groves
et~al., 2011}).

\subsection{Calibración (post-estratificación /
raking)}\label{calibraciuxf3n-post-estratificaciuxf3n-raking}

Calibrar por ola a \textbf{totales externos fiables} (p.~ej.,
\textbf{Total} y \textbf{Sexo} de la población en pobreza del año de
referencia):

\[
w_i^{(t)} = w_i^{(0)}\times w_i^{(\text{NR},t)}\times g_{i,t},\qquad
\sum_{i\in c} w_i^{(t)}x_i = T_c.
\]

\subsection{Peso longitudinal (intersección
1--3)}\label{peso-longitudinal-intersecciuxf3n-13}

Aplicar ajustes de \textbf{permanencia} análogos a la no respuesta y
\textbf{calibrar} la submuestra intersección 1--3 a totales de
referencia (p.~ej., sexo/edad de Ola 1).

\section{Operación y calidad de
medición}\label{operaciuxf3n-y-calidad-de-mediciuxf3n}

\begin{itemize}
\tightlist
\item
  Mantener \textbf{modo} e \textbf{instrumentos} constantes entre olas;
  documentar cambios inevitables y su impacto.
\item
  Protocolos de \textbf{edición/imputación} reproducibles (principios
  Fellegi--Holt), alineados con el diseño de captura
  (\citeproc{ref-groves_survey_2011}{Groves et~al., 2011}).
\end{itemize}

\subsection{Viabilidad y pertinencia del marco de
captación}\label{viabilidad-y-pertinencia-del-marco-de-captaciuxf3n}

Para la recogida de datos se utilizará un panel online y ya se ha
contactado con Netquest, que apoyó el reclutamiento en el estudio previo
y ofrece un servicio de calidad que garantiza respuestas fiables por
parte de los participantes, contando además con un panel suficientemente
amplio para cubrir los tamaños por ola previstos.

Dentro de este panel, trabajaremos con la
\href{https://www.aimc.es/otros-estudios-trabajos/clasificacion-socioeconomica/}{clasificación
socioeconómica EGM} utilizada por Netquest, que es un índice de ingreso
del hogar construido mediante un modelo de regresión que integra tres
bloques informativos diferenciados: (i) ``Grupo'' del sustentador
principal, definido por la matriz Ocupación × Nivel educativo y ordenado
por ingresos medios (continúa la tradición de ``clase social'', pero
anclada en renta observada); (ii) la actividad del sustentador
(situación laboral/ocupacional vigente, distinta del ``Grupo'', por
ejemplo ocupado, desempleado, inactivo), y (iii) la estructura del hogar
mediante la matriz Tamaño del hogar × número de perceptores de ingreso
(economías de escala y capacidad de generación de renta). Previo al
modelamiento, los ingresos declarados se corrigen/imputan con la
estimación del entrevistador y los cortes monetarios de los siete tramos
se ajustan a la distribución de sueldos del Instituto Nacional de
Estadísticas. Con ello, IE2 (\textless{} €745) e IE1 (€745--€1.312)
representan, por construcción, los menores niveles de ingreso, que son
los de interés para esta encuesta.

Empíricamente, IE1 concentra en torno al 15--16\% e IE2 al 7--8\% de la
población (≈23--24\% conjunta), con mayor presencia relativa de mujeres
en los segmentos de menor ingreso. Este marco justifica focalizar la
captación en IE2 y, de ser necesario, completar con IE1, manteniendo
selección probabilística y posterior calibración por sexo y total.
Operativamente, Netquest dispone de 4.300 casos en IE1 y 1.400 en IE2
(total 5.700), stock que supera los requerimientos del diseño
conservador (Ola 1 ≈ 1.752, Ola 3 ≈ 1.226) y respalda la viabilidad
online del estudio con margen de error ≤ ±5 p.p. por sexo y poder ≥ 0,80
para detectar diferencias de ≈8 p.p. entre sexos, sujeto a la
financiación adecuada para incentivos y logística de campo.

\section{Referencias}\label{referencias}

\phantomsection\label{refs}
\begin{CSLReferences}{1}{0}
\bibitem[\citeproctext]{ref-groves_survey_2011}
Groves, R. M., Fowler, F. J., Couper, M., Lepkowski, J. M., Singer, E.,
\& Tourangeau, R. (2011). \emph{Survey {Methodology}} (2nd ed
(Online-Ausg.)). Somerset: Wiley.

\bibitem[\citeproctext]{ref-lynn_methodology_2009}
Lynn, P. (Ed.). (2009). \emph{Methodology of {Longitudinal Surveys}}.
New York, NY: John Wiley \& Sons.

\end{CSLReferences}




\end{document}
